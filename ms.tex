\documentclass{emulateapj} 
 
\usepackage[dvipdf]{epsfig} 
\usepackage[dvips]{rotating}
\usepackage{subfigure}

\newcommand{\IRAS}{{\it IRAS}}
\newcommand{\HERSCHEL}{{\it Herschel}}
\newcommand{\SPITZER}{{\it Spitzer}}
\newcommand{\WISE}{{\it WISE}}

\bibpunct{(}{)}{;}{a}{}{,} 
 
\shorttitle{WISE/NEOWISER Coadds}
 
\shortauthors{Meisner et al.} 
 
\begin{document} 
\title{Full-depth Coadds of the WISE and First-year NEOWISE-Reactivation Images}

\author{Aaron M. Meisner\altaffilmark{1,2}}
\author{Dustin Lang\altaffilmark{3}}
\author{David J. Schlegel\altaffilmark{2}}

\altaffiltext{1}{Berkeley Center for Cosmological Physics, Berkeley, CA 94720, USA}
\altaffiltext{2}{Lawrence Berkeley National Laboratory, Berkeley, CA, 94720, USA}
\altaffiltext{3}{Department of Astronomy \& Astrophysics and Dunlap Institute, University of Toronto, Toronto, ON M5S 3H4, Canada}

\begin{abstract} 
Thanks to the Wide-field Infrared Survey Explorer (WISE) mission's
NEOWISE-Reactivation (NEOWISER) program, an entire year of post-reactivation W1
and W2 exposures were made publicly available in 2015. This data set consists of
$\sim$2.5 million exposures in each band, effectively doubling the amount
of WISE imaging available at 3.4$\mu$m and 4.6$\mu$m relative to the AllWISE
release. We have created the first ever full-sky set of coadds combining all 
publicly available W1 and W2 exposures from both the AllWISE and 
NEOWISER mission phases. We employ an adaptation of the unWISE image coaddition
framework \citep{lang14}, which preserves the native WISE angular resolution 
and is optimized for forced photometry. By incorporating two additional scans
of the entire sky, we not only improve the W1/W2 depths, but also largely
eliminate time-dependent artifacts such as off-axis moonlight contamination.
We anticipate that our new coadds will have a broad range of 
applications, including target selection for upcoming spectroscopic cosmology 
surveys, discovery of massive galaxy clusters at high redshift, and 
identification of high-redshift quasars. In particular, our full-depth 
WISE+NEOWISER coadds will be an important input for the Dark Energy 
Spectroscopic Instrument (DESI) selection of luminous red galaxy and quasar 
targets. Our full-depth W1/W2 coadds are already in use within the 
DECam Legacy Survey (DECaLS) and Mayall z-band Legacy Survey (MzLS) reduction 
pipelines. Much more work still remains in order to fully leverage NEOWISER 
imaging for astrophysical applications beyond the solar system.
\end{abstract}  
 
\keywords{methods: data analysis -- surveys -- techniques: image processing} 

\section{Introduction}

The Wide-field Infrared Survey Explorer \cite[WISE; ][]{wright10} has performed
a full-sky imaging survey in four broad mid-infrared bandpasses centered at 
3.4, 4.6, 12 and 22 microns, labeled W1-W4 from blue to red. WISE has 
dramatically enhanced our knowledge of the mid-infrared sky, and publicly 
released numerous catalog and imaging data products of high value to the 
astronomical community.

More specifically, WISE launched in December 2009, and undertook a seven 
month, full-sky survey in all of W1-W4 from February 2010 through August
2010. In September 2010, the solid hydrogen cryogen used to cool the W3 and W4 
instrumentation was depleted, significantly reducing the quality of W3 
imaging and rendering W4 unusable. Nevertheless, WISE continued surveying 
the sky from September 2010 to February 2011 in W1 and W2, including 
a portion of the mission referred to as NEOWISE \citep{neowise}. In February 
2011 WISE was placed in hibernation. In October 2013, WISE was reactivated, and
recommenced surveying the sky in W1 and W2. This W1/W2 survey is referred to as 
NEOWISE-Reactivation (NEOWISER) and is expected to continue until 2017.
Importantly, the NEOWISER data are of very nearly the same high quality as
pre-hibernation W1/W2 WISE imaging \citep{neowiser}.

Several data products consisting of full-sky, stacked WISE imaging
are currently available for the first 13 months of data.
The WISE team has created a set
of ``Atlas'' coadds smoothed by the point spread function (PSF)
using the first 7 months of data (the All-Sky release), and the first
13 months of data (the AllWISE release).
In an independent processing effort, \cite{lang14} has 
produced custom ``unWISE'' stacks analogous to the AllWISE Atlas images,
but at the full spatial resolution of the instrument. These unWISE
stacks are optimized for forced photometry, and have proven to be 
an important input for eBOSS target selection 
\citep{lang14b, eboss_qso, eboss_lrg}.

However, until now, no full-sky set of W1/W2 coadds combining
all pre and post reactivation exposures have existed. The primary motivation
for such a data product is the enhanced infrared depths achieved, which
will improve the utility of WISE in selecting higher-redshift 
spectroscopic targets, in particular for the upcoming
Dark Energy Spectroscopic Instrument \citep[DESI,][]{desi}. Additionally, 
folding in two additional scans of WISE data at each sky location allows 
time-dependent artifacts to be nulled, largely eliminating spatial 
nonuformities in image quality and derivative catalogs.

Here we present a new set of full-sky coadds generated by
combining all publicly available W1/W2 exposures from the AllWISE and NEOWISER
programs. These `full-depth' coadds are publicly available 
online\footnote{http://unwise.me}. In $\S$1 we briefly describe the W1/W2 
single-exposure data set on which our full-depth coadds are based. In $\S$2
we describe an empirical, custom photometric calibration we derived in order 
to combine pre and post reactivation WISE images. In $\S$3 we describe our
rejection of time-dependent artifacts, particularly off-axis Moon 
contamination. In $\S$4 we describe our strategy for recovering 
Moon-contaminated exposures. In $\S$5 we highlight some important aspects 
of the full-sky set of full-depth coadds generated by our processing. In 
$\S$6, we present a preliminary investigation of the improvements in WISE 
depth which have resulted from doubling the amount of W1/W2 imaging. We 
conclude in $\S$7 with a brief discussion of future uses of our present 
WISE+NEOWISER coadds, as well as remaining work to be done with existing and 
future NEOWISER imaging.

% mention DECaLS in intro and add footnote to legacysurvey website

%In the remainder of this proposal, we will refer to the data from the
%first 13 months of the mission as simply ``WISE'', in contrast to
%data from the current ``NEOWISER'' mission phase.

\acknowledgments{
We thank Roc Cutri for his guidance in making use of the WISE data products. 
}

\bibliographystyle{apj}
\bibliography{fulldepth_neo1.bib}

\end{document}
