\documentclass{emulateapj} 
 
\usepackage[dvipdf]{epsfig} 
\usepackage[dvips]{rotating}
\usepackage{subfigure}

\newcommand{\IRAS}{{\it IRAS}}
\newcommand{\HERSCHEL}{{\it Herschel}}
\newcommand{\SPITZER}{{\it Spitzer}}
\newcommand{\WISE}{{\it WISE}}

\bibpunct{(}{)}{;}{a}{}{,} 
 
\shorttitle{WISE/NEOWISER Coadds}
 
\shortauthors{Meisner et al.} 
 
\begin{document} 
\title{Full-depth Coadds of the WISE and First-year NEOWISE-Reactivation Images}

\author{Aaron M. Meisner\altaffilmark{1,2}}
\author{Dustin Lang\altaffilmark{3}}
\author{David J. Schlegel\altaffilmark{2}}

\altaffiltext{1}{Berkeley Center for Cosmological Physics, Berkeley, CA 94720, 
USA}
\altaffiltext{2}{Lawrence Berkeley National Laboratory, Berkeley, CA, 94720, 
USA}
\altaffiltext{3}{Department of Astronomy \& Astrophysics and Dunlap Institute, 
University of Toronto, Toronto, ON M5S 3H4, Canada}

\begin{abstract} 
Thanks to the Wide-field Infrared Survey Explorer (WISE) mission's
NEOWISE-Reactivation (NEOWISER) program, an entire year of post-reactivation W1
and W2 exposures were made publicly available in 2015. This data set consists of
$\sim$2.5 million exposures in each band, effectively doubling the amount
of WISE imaging available at 3.4$\mu$m and 4.6$\mu$m relative to the AllWISE
release. We have created the first ever full-sky set of coadds combining all 
publicly available W1 and W2 exposures from both the AllWISE and 
NEOWISER mission phases. We employ an adaptation of the unWISE image coaddition
framework \citep{lang14}, which preserves the native WISE angular resolution 
and is optimized for forced photometry. By incorporating two additional scans
of the entire sky, we not only improve the W1/W2 depths, but also largely
eliminate time-dependent artifacts such as off-axis moonlight contamination.
We anticipate that our new coadds will have a broad range of 
applications, including target selection for upcoming spectroscopic cosmology 
surveys, discovery of distant/massive galaxy clusters, and 
identification of high-redshift quasars. In particular, our full-depth 
WISE+NEOWISER coadds will be an important input for the Dark Energy 
Spectroscopic Instrument (DESI) selection of luminous red galaxy and quasar 
targets. Our full-depth W1/W2 coadds are already in use within the 
DECam Legacy Survey (DECaLS) and Mayall z-band Legacy Survey (MzLS) reduction 
pipelines. Much more work still remains in order to fully leverage NEOWISER 
imaging for astrophysical applications beyond the solar system.
\end{abstract}  
 
\keywords{methods: data analysis -- surveys -- techniques: image processing} 

\section{Introduction}

The Wide-field Infrared Survey Explorer \cite[WISE; ][]{wright10} has performed
a full-sky imaging survey in four broad mid-infrared bandpasses centered at 
3.4, 4.6, 12 and 22 microns, labeled W1-W4 from blue to red. WISE has 
dramatically enhanced our knowledge of the mid-infrared sky, and publicly 
released numerous catalog and imaging data products of high value to the 
astronomical community.

More specifically, WISE launched in December 2009, and undertook a seven 
month, full-sky survey in all of W1-W4 from February 2010 through August
2010. In September 2010, the solid hydrogen cryogen used to cool the W3 and W4 
instrumentation was depleted, significantly reducing the quality of W3 
imaging and rendering W4 unusable. Nevertheless, WISE continued surveying 
the sky from September 2010 to February 2011 in W1 and W2, including 
a portion of the mission referred to as NEOWISE \citep{neowise}. In February 
2011 WISE was placed in hibernation. In October 2013, WISE was reactivated, and
recommenced surveying the sky in W1 and W2. This W1/W2 survey is referred to as 
NEOWISE-Reactivation (NEOWISER) and is expected to continue until 2017.
Importantly, the NEOWISER data are of very nearly the same high quality as
pre-hibernation W1/W2 WISE imaging \citep{neowiser}.

Several data products consisting of full-sky, stacked WISE imaging are 
currently available for the first 13 months of data. The WISE team has created 
a set of ``Atlas'' coadds smoothed by the point spread function (PSF) using the
first 7 months of data (the All-Sky release), and the first 13 months of data 
(the AllWISE release). In an independent processing effort, \cite{lang14} has 
produced custom ``unWISE'' stacks analogous to the AllWISE Atlas images, but at
the full spatial resolution of the instrument. These unWISE stacks are 
optimized for forced photometry, and have proven to be an important input for 
eBOSS target selection \citep{lang14b, eboss_qso, eboss_lrg}.

However, until now, no full-sky set of W1/W2 coadds combining all pre and post 
reactivation exposures has existed. The primary motivation for such a data 
product is the enhanced infrared depths achieved, which, among
other benefits, will improve the utility of WISE in selecting higher-redshift 
spectroscopic targets, in particular for the upcoming Dark Energy Spectroscopic
Instrument \citep[DESI,][]{desi}. Additionally, folding in two additional scans
of WISE data at each sky location allows time-dependent artifacts to be nulled,
largely eliminating spatial nonuformities in image quality and derivative 
catalogs.

% add sentence to above paragraph about Faherty arxiv survey saying
% WISE+NEOWISER coadds are highly valuable ?

Here we present a new set of full-sky coadds generated by
combining all publicly available W1/W2 exposures from the AllWISE and NEOWISER
programs, using an adaptation of the \cite{lang14} unWISE methodology. These 
`full-depth' coadds are publicly available online\footnote{http://unwise.me}.

In $\S$\ref{sec:data} we briefly describe the W1/W2 single-exposure data set 
from which our coadds are constructed. In $\S$\ref{sec:coadd}, we review the 
important aspects of the unWISE coaddition framework we employ and list the 
processing steps which are newly introduced in this work. In 
$\S$\ref{sec:calib} we describe an empirical, custom photometric calibration we
derived in order to combine pre and post reactivation WISE images. In 
$\S$\ref{sec:moon} we describe our rejection of time-dependent artifacts, 
particularly off-axis Moon contamination. In $\S$\ref{sec:recover} we describe
our procedure for recovering Moon-contaminated exposures. In 
$\S$\ref{sec:results} we highlight some important aspects of the full-sky set 
of full-depth coadds generated by our processing. In $\S$\ref{sec:depth}, we 
present a preliminary investigation of the improvements in WISE depth which 
have resulted from doubling the amount of W1/W2 imaging. We conclude in 
$\S$\ref{sec:future} with a brief discussion of future uses of our present 
WISE+NEOWISER coadds, as well as remaining work to be done with existing and 
future NEOWISER imaging.

\section{Data}
\label{sec:data}

The 1016 pixel by 1016 pixel WISE single-exposure ``L1b'' images represent
the input data for our W1/W2 coadds. Specifically, for each L1b frameset, we 
make use of the per-band \verb|-int-|, \verb|-msk-| and \verb|-unc-| images, 
which respectively give the measured sky intensity and associated per-pixel 
bitmask values and uncertainty estimates. We have obtained a local copy of 
these files for every W1 and W2 frameset, including those from the AllWISE 
release ($\sim$2.9M framesets) and first-year NEOWISER release ($\sim$2.5M 
framesets). In total we have analyzed $\sim$5.4M framesets in each of W1 and 
W2, corresponding to a total of $\sim$32M L1b image files and $\sim$71TB of 
pixel data.

% would be fun to say how many pixels of input there are total

In addition to L1b images, we also make use of several catalog-level WISE
data products. During the photometric calibration described in 
$\S$\ref{sec:calib}, we select sources to photometer based on the AllWISE 
Source Catalog \citep{cutri13}. Also, to flag and reject bright solar system 
planets ($\S$\ref{sec:moon}), we employ the WISE Known Solar System Object
Possible Association List for each mission phase 
\citep{cutri12, cutri13,cutri15}.

\section{Image Coaddition Methodology}
\label{sec:coadd}

% should this section go before the ``data'' section ??

To coadd the W1/W2 single exposures, we make use of the \cite{lang14}
unWISE coaddition framework, and perform our image processing with an 
adaptation of the codebase from that work. We briefly mention
a few of the salient aspects of the unWISE coaddition methodology here; for
a full discussion see \cite{lang14}.

% mention that we use the same tile centers as Atlas and original unWISE
Like the official WISE Atlas coadds, unWISE processing divides the sky
into a set of 18240 $\sim$1.5$^{\circ}$$\times$$\sim$1.5$^{\circ}$ tiles
arranged along iso-declination rings. Unlike the Atlas tiles, the
unWISE code uses Lanczos interpolation to preserve the native WISE 
angular resolution during coaddition, creating stacked outputs which are 2048 
pixels on a side, with pixels of size 2.75$''$.

During the course of this work, various modifications have been made to the
 unWISE codebase and methodology. Here we highlight the important 
updates/changes:

\begin{itemize}
\item We include all publicly available NEOWISER W1 and W2 exposures, 
approximately doubling the number of input L1b frames relative to \cite{lang14}.
\item We adopt custom zero points based on repeat photometry at the ecliptic 
poles. In contrast, the \cite{lang14} unWISE processing adopted zero points 
based on the L1b header metadata.
\item We explicitly reject exposures contaminated by the Moon and/or solar 
system planets. No such rejection was included in the \cite{lang14} unWISE 
W1/W2 coadds, although these and other artifacts were addressed to some extent 
via general-purpose outlier rejection.
\item In this work we attempt to recover Moon-affected frames by applying 
polynomial background level corrections to the contaminated exposures.
\end{itemize}

The outputs generated in this work follow the same data model as those of 
\cite{lang14}. For each tile, a stacked image is created, plus auxiliary
maps of useful quantities such as the per-pixel inverse variance and integer 
coverage. Like those of \cite{lang14}, our coadds have units of Vega 
nanomaggies.

\section{Custom Photometric Calibration}
\label{sec:calib}
In order to combine frames across multiple mission phases, it is necessary
to place all exposures on a common photometric calibration such that
the multiplicative scalings of all images are consistent. Each L1b
image includes a \verb|MAGZP| header keyword which gives the official
Vega zero-point of that exposure. These zero-points are essentially
predictions of system throughput based on predictors such as beam splitter
assembly temperature, and have been shown to differ by up to several percent 
relative to gain variations measured empirically with single-exposure
photometry of calibrator sources \citep{cutri13, cutri15}.

We therefore sought to achieve an empirical relative photometric calibration
across all mission phases accurate at the several mmag level. To do so, we 
analyzed repeat measurements of compact sources near the 
ecliptic poles, where WISE has gathered data every $\sim$$95$ minutes 
throughout the entire mission. Specifically, our sample consists of moderately 
bright, unsaturated compact sources with $|\beta| > 85^{\circ}$, avoiding a 
wedge defined by $-90^{\circ}$$<$$\lambda$$<25^{\circ}$ near the south ecliptic
pole to exclude the LMC. The positions and nominal average magnitudes of our 
moderately bright source sample were drawn from the AllWISE Source Catalog, 
selecting sources with 10.6 $<$ \verb|w1mpro| $<$ 13.1 and 9.2 $<$ 
\verb|w2mpro| $<$ 11.7 and with \verb|w?cc_map|=0 in the band of interest.

These spatial, magnitude and flag cuts yield samples of $\sim$109,000 
($\sim$27,000) unique calibrator sources in W1 (W2). We perform aperture 
photometry using \verb|djs_phot| with a 27.5$''$ radius for each calibrator 
source in every L1b exposure in which it appears sufficiently far from the 
image boundary. This results in a catalog of $\sim$45M ($\sim$15M) W1 (W2) 
single-epoch aperture fluxes (in units of DN), which will form the basis for 
our derived time-dependent zero points. The typical calibrator source 
contributes $\gtrsim$400 epochs of photometry.

We desired a photometic calibration with time resolution of one day. To
achieve this, we grouped our single-exposure photometric measurements in two 
ways. First, for each 
unique source, we lumped all of its All-Sky phase measurements together to 
obtain its median flux (DN) in our aperture photometry system during this 
phase. This is justified because the All-Sky release photometric zero points 
are known to be remarkably stable \citep{jarrett11}. Indeed, using our ecliptic
pole singe-exposure photometry database, we were able to confirm that the 
All-Sky phase zero point was stable at the $\le$2 mmag level in each band. 
Next, for every aperture flux after the All-Sky phase, we calculated a 
multiplicative enhancement factor
implied by the ratio of that measurement to the appropriate source's
median All-Sky phase flux. We then grouped these 
multiplicative enhancement factors into one-day bins, and quote the median per 
bin as the change in multiplicative
image scaling relative to the All-Sky zero point. For the  
All-Sky phase zero points, we adopted the \verb|MAGZP| values of 20.752 in W1 
and 19.596 in W2.

Figure 1 shows our derived zero points for each WISE mission phase as 
compared to the \verb|MAGZP| values obtained from the L1b headers. When
using our custom zero points to scale L1b images, we employ an 
interpolation scheme based on a series of tapered polynomials and 
error functions, meant to smooth out noise in our per-day zero point
determinations. In general, our per-day zero points agree reasonably
well with the \verb|MAGZP| values, although there are often 
percent-level differences, and at times disagreement at the several percent
level. In some cases the time-trends within a particular mission phase
show qualitative disagreement (e.g. both W1 and W2 during the NEOWISE
mission phase).

% do i want to mention comparison between ecliptic north and south poles ??

% caption of zero point figure needs to clarify the '4band', '3band', '2band' ..
% terminology

\section{Removing Time-dependent Artifacts}
\label{sec:moon}

%discuss moon first and focus on moon
The WISE scan strategy is such that a typical sky location will be 
observed at approximately six month intervals, with each six monthly
``visit'' yielding a series of $\sim$12 exposures within a time span of $\sim$1
day. During the AllWISE release, most of the sky experienced just two visits
of W1/W2 imaging. Incorporating exposures from both the AllWISE and
NEOWISER phases effectively doubles this value to four visits everywhere
on the sky. If we coadded the AllWISE+NEOWISER data naively, without
concern for time dependent artifacts, we would risk corrupting regions of the 
sky that were pristine during the AllWISE phase. We have found that by 
leveraging the added redundancy of extra NEOWISER visits while carefully 
addressing time-dependent artifacts, we can create full-depth coadds which are 
nearly artifact-free over the entire sky.

The dominant time-dependent artifact in W1/W2 images is off-axis
scattered light from the Moon, which can significantly contaminate
images at angular separations of up to many tens of degrees. This
contamination manifests itself in L1b exposures as a strongly spatially 
variable background level, which in certain images can be smooth, but in
others can show a very complex morphology.

In the \cite{lang14} unWISE processing, no steps were taken specifically
to mitigate Moon contamination in W1 and W2. However \cite{lang14} did 
address Moon contamination in W3 and W4. \cite{lang14}
inspected all exposures flagged with the \verb|MOON_MASKED| bit, and 
discarded those frames with abnormally large pixel value standard deviations,
indicative of a strongly varying background level (see \cite{lang14} $\S$2
for full details). In the present work we have applied this same
Moon rejection criterion to W1 and W2 frames. Because of the added redundancy
of two extra NEOWISER scans, we are thus able to reject many W1/W2
frames which are contaminated by moonlight, while still retaining 
sufficient uncontaminated coverage everywhere on the sky to avoid leaving
any holes in the stacks.

Figure 2 shows the dramatic improvement achieved toward maintaining a 
consistent coadd background level by virtue of folding in NEOWISER data and 
applying the frame-level Moon rejection cut, for a tile with severe Moon 
contamination during the AllWISE phase.

A second, less common form of time-dependent artifact results when
bright solar system planets (Mars, Jupiter and Saturn) pass through
the WISE field of view. These planet sightings are prominent in the 
\cite{lang14} unWISE stacks, as no steps were taken to address these
occurrences. In constructing our new full-depth coadds, we have used the Known 
Solar System Possible Association list to identify all exposures in which
Mars, Jupiter or Saturn falls within the WISE field of view. We discard
such frames during coaddition. 

These bright planets also result in scattered light halos a few degrees in size.
Therefore, we additionally use ephemerides to identify all frames within 
2.5$^{\circ}$ of these planets. During the standard unWISE coaddition procedure,
such frames are discarded. However, we later attempt to recover these
frames according to the procedure described below in $\S$\ref{sec:recover}.

% figure showing single-tile example of removed moon contamination

\section{Recovering Contaminated Frames}
\label{sec:recover}

Although we were able to dramatically reduce the impact of Moon contamination
on our coadds with the exposure rejection procedure of $\S$\ref{sec:moon},
we would ideally like to recover as much Moon-contaminated data as possible,
rather than simply discard it all outright. To that end, we have added an 
after-burner step to our coaddition procedure which attempts to salvage those 
frames which were flagged with \verb|MOON_MASKED| and displayed abnormally 
large pixel value standard deviations. The procedure we employ is a variant of 
that described in $\S6.4.1$ of \cite{meisner14}.

The first two rounds of unWISE coaddition still proceed exactly as 
described in \cite{lang14}. These steps yield a Moon-free stack which we 
subsequenty use as a reference image to compare against, and derive low-order 
corrections for, each Moon-contaminated frame.

For each frame initially rejected on the basis of Moon contamination, we 
first resample the exposure onto the coadd astrometry. We then divide
the exposure into quadrants which we analyze separately. For each quadrant,
we will attempt to model the Moon contamination with a polynomial 
offset as a function of L1b $x$, $y$ pixel coordinates. We begin by masking out 
the brightest and faintest 5\% of pixels in the reference coadd, since pixels
with bright compact sources will not be very informative for background level
modeling. We then fit the difference between the masked L1b quadrant and masked
reference coadd with a fourth order polynomial in L1b $x$, $y$ coordinates. We 
evaluate the chi-squared of this model, using the reference coadd's per-pixel
standard deviation to construct per-pixel uncertainty estimates.

For each quadrant, we deem the polynomial correction to be a satisfactory 
description of the scattered moonlight if the mean per-pixel chi-squared is 
less than 2.5. In that case, we then subtract the polynomial correction from 
the quadrant, and consider the quadrant ``recovered''. Quadrants with poor 
chi-squared are discarded and remain excluded from the coadd. Once a list of 
all recovered quadrants has been assembled, these are accumulated into the 
existing reference coadd to produce a final set of outputs for the tile under 
consideration.

Figure 3 provides an illustration of our polynomial background modeling
procedure. We were able to recover 54\% of Moon-contaminated data
in W1 and 33\% in W2. We also apply this polynomial
correction procedure to frames that we flagged as potentially affected
by scattered light halos from bright solar system planets.

% quadrant warping example figure

\section{Overview of Results}
\label{sec:results}

Figures 4 and 5 show large-scale renderings of our full-depth coadds 
over the north and south Galactic caps respectively. It is 
apparent that the Moon contamination has been completely eliminated in
W1 and dramatically reduced in W2. It is possible that simply folding in 
additional NEOWISER W2 frames from forthcoming data releases will diminish the 
remaining Moon imprint to such an extent that no additional processing 
modifications will be required to address this issue.

Figure 6 shows zoom-ins of low ecliptic latitude fields which illustrate
visually the reduction in statistical noise which has been achieved
by doubling the number of input exposures.

Given that we imposed various cuts to eliminate time-dependent artifacts,
it is reasonable to ask whether we have created any zero-coverage holes
as a result. We have checked all 18,240 of our integer coverage maps, and find
that every pixel has $\ge$18 epochs in W1 and $\ge$15 epochs in W2.

% figure showing before/after of SGP on large scales in both W1/W2
% figure showing close-up at low ecl latitude with reduced noise

% would be fun to say how many pixels of output there are total

\section{Preliminary Depth Characterization}
\label{sec:depth}

It is important to quantify the effect of increased coverage on
the W1/W2 depths achieved by catalogs based on our AllWISE+NEOWISER coadds 
relative to those based on the AllWISE-only coadds of \cite{lang14}. A
full characterization should systematically explore a range of 
ecliptic and galactic latitudes, to consider the full spectrum of interplay
between decreased statistical noise and confusion. For isolated sources in the 
sky noise limit, we would expect a doubling of the W1/W2 imaging data
to increase the depth in each band by 0.38 mag.

The only existing catalogs based on our AllWISE+NEOWISER W1/W2 coadds
are forced photometry catalogs corresponding to DECam Legacy
Survey\footnote{http://legacysurvey.org} (DECaLS) DR2 optical sources. 
Optical sources from DECaLS DR1 are accompanied by forced based on the
\cite{lang14} W1/W2 coadds. Therefore, comparing the S/N of W1/W2
fluxes in DR2 versus DR1 allows us to obtain a preliminary estimate of the 
increased depths in regions of low ecliptic and high galactic latitude.

We have cross-matched DECaLS DR1 and DR2 sources, and selected samples of
faint WISE detections in DR1 for each band, $5<$ (W? S/N) $<6$. We find that
in DECaLS DR2, these same sources have on average 1.33$\times$ (1.19$\times$)
higher S/N in W1 (W2). For comparison, the expectation based purely on
reduced statistical noise in the WISE imaging is 1.38$\times$ improvement in
S/N.

Much more work is needed to characterize the increased sensitivity
of WISE to faint sources due to the inclusion of NEOWISER
imaging, as this will inform forecasts about the final depths expected
upon WISE's permanent retirement.

\section{Conclusion \& Future Work}
\label{sec:future}

We have created a full-sky set of W1/W2 coadds which combine all publicly
available exposures from both the AllWISE and NEOWISER releases. Doubling the 
amount of WISE imaging relative to the AllWISE release has resulted in improved
W1/W2 depths and allowed for the elimination of nearly all time-dependent 
artifacts. Our new AllWISE+NEOWISER W1/W2 coadds are publicly available
via http://unwise.me.

Although the present analysis constituted a significant data processing 
endeavor, it represents only a small fraction of the work that must
be done to maximize science return from the NEOWISER imaging data set.
Creation of time-resolved coadds spanning the $\sim$5 year WISE-NEOWISER
time baseline would enable a wealth of important time-domain projects,
ranging from brown dwarf searches to quasar variability. As
additional years worth of NEOWISER data become publicly available, it
will be necessary to continue updating the full-depth WISE+NEOWISER coadds
to maximize the achieved depth. Finally, the creation WISE-only (as opposed
to forced photometry) catalogs based on deep coadds which include
NEOWISER images will be needed in order to enable a variety of exciting
discoveries.

\acknowledgments{
We thank Roc Cutri for his guidance in making use of the WISE data products. 
We thank Stephen Bailey for downloading the L1b images to NERSC. We thank
Doug Finkbeiner for providing access to the big-memory computers which
were used to run coadds near the ecliptic poles. We thank Peter Nugent for 
providing HPC assistance. We thank many members of the DECaLS/MzLS data team for
valuable feedback, especially Arjun Dey, Doug Finkbeiner, John Moustakas
and Bob Blum.

This research used resources of the National Energy Research Scientific 
Computing Center, a DOE Office of Science User Facility supported by the Office
of Science of the U.S. Department of Energy under Contract No. 
DE-AC02-05CH11231.

This research made use of the NASA Astrophysics Data System (ADS) and the IDL 
Astronomy User's Library at Goddard. \footnote{Available at 
\texttt{http://idlastro.gsfc.nasa.gov}}

This research makes use of data products from the Wide-field Infrared
Survey Explorer, which is a joint project of the University of California, Los
Angeles, and the Jet Propulsion Laboratory/California Institute of Technology,
funded by the National Aeronautics and Space Administration. This research
also makes use of data products from NEOWISE, which is a project of the Jet
Propulsion Laboratory/California Institute of Technology, funded by the
Planetary Science Division of the National Aeronautics and Space
Administration. }

\bibliographystyle{apj}
\bibliography{fulldepth_neo1.bib}

% cite software packages such as numpy, scipy, matplotlib ??
% mention idlutils

\end{document}
