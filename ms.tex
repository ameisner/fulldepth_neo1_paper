\documentclass{emulateapj} 
 
\usepackage[dvipdf]{epsfig} 
\usepackage[dvips]{rotating}
\usepackage{subfigure}

\newcommand{\IRAS}{{\it IRAS}}
\newcommand{\HERSCHEL}{{\it Herschel}}
\newcommand{\SPITZER}{{\it Spitzer}}
\newcommand{\WISE}{{\it WISE}}

\bibpunct{(}{)}{;}{a}{}{,} 
 
\shorttitle{WISE/NEOWISER Coadds}
 
\shortauthors{Meisner et al.} 
 
\begin{document} 
\title{Full-depth Coadds of the WISE and First-year NEOWISE-Reactivation Images}

\author{Aaron M. Meisner\altaffilmark{1,2}}
\author{Dustin Lang\altaffilmark{3}}
\author{David J. Schlegel\altaffilmark{2}}

\altaffiltext{1}{Berkeley Center for Cosmological Physics, Berkeley, CA 94720, 
USA}
\altaffiltext{2}{Lawrence Berkeley National Laboratory, Berkeley, CA, 94720, 
USA}
\altaffiltext{3}{Department of Astronomy \& Astrophysics and Dunlap Institute, 
University of Toronto, Toronto, ON M5S 3H4, Canada}

\begin{abstract} 
Thanks to the Wide-field Infrared Survey Explorer (WISE) mission's
NEOWISE-Reactivation (NEOWISER) program, an entire year of post-reactivation W1
and W2 exposures were made publicly available in 2015. This data set consists of
$\sim$2.5 million exposures in each band, effectively doubling the amount
of WISE imaging available at 3.4$\mu$m and 4.6$\mu$m relative to the AllWISE
release. We have created the first ever full-sky set of coadds combining all 
publicly available W1 and W2 exposures from both the AllWISE and 
NEOWISER mission phases. We employ an adaptation of the unWISE image coaddition
framework \citep{lang14}, which preserves the native WISE angular resolution 
and is optimized for forced photometry. By incorporating two additional scans
of the entire sky, we not only improve the W1/W2 depths, but also largely
eliminate time-dependent artifacts such as off-axis moonlight contamination.
We anticipate that our new coadds will have a broad range of 
applications, including target selection for upcoming spectroscopic cosmology 
surveys, discovery of massive galaxy clusters at high redshift, and 
identification of high-redshift quasars. In particular, our full-depth 
WISE+NEOWISER coadds will be an important input for the Dark Energy 
Spectroscopic Instrument (DESI) selection of luminous red galaxy and quasar 
targets. Our full-depth W1/W2 coadds are already in use within the 
DECam Legacy Survey (DECaLS) and Mayall z-band Legacy Survey (MzLS) reduction 
pipelines. Much more work still remains in order to fully leverage NEOWISER 
imaging for astrophysical applications beyond the solar system.
\end{abstract}  
 
\keywords{methods: data analysis -- surveys -- techniques: image processing} 

\section{Introduction}

The Wide-field Infrared Survey Explorer \cite[WISE; ][]{wright10} has performed
a full-sky imaging survey in four broad mid-infrared bandpasses centered at 
3.4, 4.6, 12 and 22 microns, labeled W1-W4 from blue to red. WISE has 
dramatically enhanced our knowledge of the mid-infrared sky, and publicly 
released numerous catalog and imaging data products of high value to the 
astronomical community.

More specifically, WISE launched in December 2009, and undertook a seven 
month, full-sky survey in all of W1-W4 from February 2010 through August
2010. In September 2010, the solid hydrogen cryogen used to cool the W3 and W4 
instrumentation was depleted, significantly reducing the quality of W3 
imaging and rendering W4 unusable. Nevertheless, WISE continued surveying 
the sky from September 2010 to February 2011 in W1 and W2, including 
a portion of the mission referred to as NEOWISE \citep{neowise}. In February 
2011 WISE was placed in hibernation. In October 2013, WISE was reactivated, and
recommenced surveying the sky in W1 and W2. This W1/W2 survey is referred to as 
NEOWISE-Reactivation (NEOWISER) and is expected to continue until 2017.
Importantly, the NEOWISER data are of very nearly the same high quality as
pre-hibernation W1/W2 WISE imaging \citep{neowiser}.

Several data products consisting of full-sky, stacked WISE imaging are 
currently available for the first 13 months of data. The WISE team has created 
a set of ``Atlas'' coadds smoothed by the point spread function (PSF) using the
first 7 months of data (the All-Sky release), and the first 13 months of data 
(the AllWISE release). In an independent processing effort, \cite{lang14} has 
produced custom ``unWISE'' stacks analogous to the AllWISE Atlas images, but at
the full spatial resolution of the instrument. These unWISE stacks are 
optimized for forced photometry, and have proven to be an important input for 
eBOSS target selection \citep{lang14b, eboss_qso, eboss_lrg}.

However, until now, no full-sky set of W1/W2 coadds combining all pre and post 
reactivation exposures have existed. The primary motivation for such a data 
product is the enhanced infrared depths achieved, which will improve the 
utility of WISE in selecting higher-redshift spectroscopic targets, in 
particular for the upcoming Dark Energy Spectroscopic Instrument 
\citep[DESI,][]{desi}. Additionally, folding in two additional scans of WISE 
data at each sky location allows time-dependent artifacts to be nulled, largely
eliminating spatial nonuformities in image quality and derivative catalogs.

Here we present a new set of full-sky coadds generated by
combining all publicly available W1/W2 exposures from the AllWISE and NEOWISER
programs. These `full-depth' coadds are publicly available 
online\footnote{http://unwise.me}. In $\S$\ref{sec:data} we briefly describe 
the W1/W2 single-exposure data set on which our full-depth coadds are based. 
In $\S$\ref{sec:calib} we describe an empirical, custom photometric calibration
we derived in order to combine pre and post reactivation WISE images. In 
$\S$\ref{sec:moon} we describe our rejection of time-dependent artifacts, 
particularly off-axis Moon contamination. In $\S$\ref{sec:recover} we describe
our strategy for recovering Moon-contaminated exposures. In 
$\S$\ref{sec:results} we highlight some important aspects of the full-sky set 
of full-depth coadds generated by our processing. In $\S$\ref{sec:depth}, we 
present a preliminary investigation of the improvements in WISE depth which 
have resulted from doubling the amount of W1/W2 imaging. We conclude in 
$\S$\ref{sec:future} with a brief discussion of future uses of our present 
WISE+NEOWISER coadds, as well as remaining work to be done with existing and 
future NEOWISER imaging.

\section{Data}
\label{sec:data}

The 1016 pixel by 1016 pixel WISE single-exposure ``L1b'' images form the basis
for our W1/W2 coadds. Specifically, for each L1b frameset, we make use of the 
per-band \verb|-int-|, \verb|-msk-| and \verb|-unc-| images, which respectively
give the measured sky intensity and associated per-pixel bitmask values and 
uncertainty estimates. We have obtained a local copy of these files for every 
W1 and W2 frameset, including those from the AllWISE release ($\sim$2.9M 
framesets) and first-year NEOWISER release ($\sim$2.5M framesets). In
total we have analyzed $\sim$5.4M framesets in each of W1 and W2, corresponding
to a total of $\sim$32M L1b image files and $\sim$71TB of pixel data.

In addition to L1b images, we also make use of several official WISE
catalog-level data products. During the photometric calibration described
in $\S$\ref{sec:calib}, we select sources to photometer based on the AllWISE 
Source Catalog \citep{cutri13}. Also, to flag and reject bright solar system 
planets ($\S$\ref{sec:moon}), we make use of the WISE Known Solar System Object
Possible Association List for each mission phase 
\citep{cutri12, cutri13,cutri15}.

% mention that we use the allwise source catalog during the photometric
% mention that we also use the various ``known solar system possible 
% association list'' catalogs

\section{Custom Photometric Calibration}
\label{sec:calib}
In order to combine frames across multiple mission phases, it is necessary
to place all exposures on a common photometric calibration such that
the multiplicative scalings of all images are consistent. Each L1b
image includes a \verb|MAGZP| header keyword which gives the official
Vega zero-point of that exposure. These zero-points are essentially
predictions of system throughput based on predictors such as beam splitter
assembly temperature, and have been shown to differ by up to several percent 
from time-dependent gain variations measured empirically with single-exposure
photometry of standard sources \citep{cutri13, cutri15}.

We therefore sought to achieve an empirical relative photometric calibration
across all mission phases accurate at the several mmag level. Our basic
strategy was to analyze repeat measurements of compact sources near the 
ecliptic poles, where WISE gathers data every $\sim$$95$ minutes throughout the 
entire mission. Specifically, our sample consists of moderately bright,
unsaturated compact sources with $|\beta| > 85^{\circ}$, and avoiding a wedge 
defined by $-90^{\circ}$$<$$\lambda$$<25^{\circ}$ to exclude the LMC. The 
positions and nominal average magnitudes of our moderately bright source
sample were drawn from the AllWISE Source Catalog, selecting sources with 
10.6 $<$ \verb|w1mpro| $<$ 13.1 and 9.2 $<$ \verb|w2mpro| $<$ 11.7 and with
\verb|w?cc_map|=0 in the band of interest.

These spatial, magnitude and flag cuts yield samples of $\sim$109,000 
($\sim$27,000) unique calibrator sources in W1 (W2). We perform aperture 
photometry using \verb|djs_phot| with a 27.5$''$ radius for each calibrator 
source in every L1b exposure in which it appears sufficiently far from the 
image boundary. This results in a catalog of $\sim$45M ($\sim$15M) W1 (W2) 
single-epoch aperture fluxes (in units of DN) which form the basis for our 
derived time-dependent zero points. The typical calibrator source contributes 
$\gtrsim$400 epochs of photometry.

We sought a photometic calibration with time resolution of one day. We
grouped the photometric measurements in two ways. For each unique source,
we lumped all of its All-sky phase measurements together to obtain its median
flux (DN) in our aperture photometry system during this phase. This is
justified because the All-sky release photometric zero points are known
to be remarkably stable. Indeed, using our ecliptic pole singe-exposure 
photometry database, we were able to confirm that the All-sky phase zero point 
was stable at the $\le$2 mmag level in each band. For every aperture
flux after the All-sky phase, we calculated a gain enhancement factor
implied by that measurement relative to the appropriate sources
median All-sky phase flux. We then grouped these gain enhancement factors
into one-day bins, and quoted the median per bin as the multiplicative
change in zero point relative to the All-sky zero point. We took the 
All-sky zero points (20.752 in W1 and 19.596 in W2) at face value.

Figure 1 shows our derived zero points for each WISE mission phase as 
compared to the \verb|MAGZP| values obtained from the L1b headers. When
using our custom zero points to scale L1b images, we employ an 
interpolation scheme based on a series of tapered polynomials and 
error functions, meant to smooth out noise in our per-day zero point
determinations. In general, our per-day zero points agree reasonably
well with the \verb|MAGZP| values, although there are often 
percent-level differences, and at times disagreement at the several percent
level. In some cases the time-trends within a particular mission phase
show qualitative disagreement (e.g. both W1 and W2 during the NEOWISE
mission phase).

% do i want to mention comparison between ecliptic north and south poles ??

% caption of zero point figure needs to clarify the '4band', '3band', '2band' ..
% terminology

\section{Removing Time-dependent Artifacts}
\label{sec:moon}

\section{Recovering Contaminated Frames}
\label{sec:recover}

\section{Overview of Results}
\label{sec:results}

\section{Preliminary Depth Characterization}
\label{sec:depth}

\section{Conclusion \& Future Work}
\label{sec:future}

% add footnote to legacysurvey website in section that compares
% wise depths in DECaLS DR1 versus DR2


\acknowledgments{
We thank Roc Cutri for his guidance in making use of the WISE data products. 

This research made use of the NASA Astrophysics Data System (ADS) and the IDL 
Astronomy User's Library at Goddard. \footnote{Available at 
\texttt{http://idlastro.gsfc.nasa.gov}}

This research makes use of data products from the Wide-field Infrared
Survey Explorer, which is a joint project of the University of California, Los
Angeles, and the Jet Propulsion Laboratory/California Institute of Technology,
funded by the National Aeronautics and Space Administration. This publication
also makes use of data products from NEOWISE, which is a project of the Jet
Propulsion Laboratory/California Institute of Technology, funded by the
Planetary Science Division of the National Aeronautics and Space
Administration. }

\bibliographystyle{apj}
\bibliography{fulldepth_neo1.bib}

% cite software packages such as numpy, scipy, matplotlib ??
% mention idlutils

\end{document}
